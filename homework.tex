\documentclass[12pt]{article}

\usepackage{fouriernc}
\usepackage{amssymb}
\usepackage{amsmath}
\usepackage{amsfonts}
\usepackage[utf8]{inputenc}
\usepackage[T1]{fontenc}
\usepackage[margin=1in]{geometry}

\newcommand{\curly}[1]{\left\{      #1 \right\}     }
\newcommand{\round}[1]{\left(       #1 \right)      }
\newcommand{\hard} [1]{\left[       #1 \right]      }
\newcommand{\abs}  [1]{\left|       #1 \right|      }
\newcommand{\floor}[1]{\left\lfloor #1 \right\rfloor}
\newcommand{\ceil} [1]{\left\lceil  #1 \right\rceil }
\newcommand{\R}    [0]{\mathbb{R}                   }
\newcommand{\Z}    [0]{\mathbb{Z}                   }

\setlength{\parskip}{1em}
\setlength{\parindent}{0in}

\title{Divisibility Homework}
\author{Tim Harding}

\begin{document}
\maketitle

\section{}

\subsection{Problem}
Given $n \in \mathbb{Z}$, prove that $3 \mid (n^3 - n)$.

\subsection{Solution}
We know that $n$ can be written in one of three ways:
\begin{enumerate}
    \item If $n = 3a$ for some $a \in \Z$, we have \begin{align*}
        n^3 &- n \\
        (3a)^3 &- 3a \\
        27a^3 &- 3a \\
        3(9a^3 &- a)
    \end{align*} which is divisible by 3.

    \item If $n = 3a + 1$ for some $a \in \Z$, we have \begin{align*}
        &n^3 - n \\
        n&(n^2 - 1) \\
        (3a + 1) &((3a + 1)^2 - 1) \\
        (3a + 1) &(9a^2 + 6a + 1 - 1) \\
        3 ((3a + 1)&(3a^2 + 2a))
    \end{align*} which is divisible by 3.

    \item If $n = 3a + 2$ for some $a \in \Z$, we have \begin{align*}
        &n^3 - n \\
        n&(n^2 - 1) \\
        (3a + 2) &((3a + 2)^2 - 1) \\
        (3a + 2) &(9a^2 + 12a + 4 - 1) \\
        3 ((3a + 2) &(3a^2 + 4a + 1))
    \end{align*} which is divisible by 3.
\end{enumerate}



\section{}

\subsection{Problem}
Given $a, b \in \Z$ such that $a$ and $b$ are odd, $b > 0$, and $b \nmid a$, prove that there exist $s, t \in \Z$ such that $a = bs + t$, $t$ is odd, and $\abs{t} < b$.

\subsection{Solution}
There exist $k, q \in \Z$ such that $a = 2k + 1$ and $b = 2q + 1$. We know from the division algorithm that there exist $s, t \in \Z$ such that $a = bs + t$ with $0 \leq t < b$. By substitution, we have
\begin{align*}
    2k + 1 &= (2q + 1)s + t \\
    t &= 2k + 1 - (2q + 1)s
\end{align*}


\section{}

\subsection{Problem}
List the integers $100!$, $100^{100}$, $2^{100}$, and $(50!)^2$ in order of increasing size with reasoning provided.

\subsection{Solution}
In order of increasing size, we have
\begin{enumerate}
    \item $2^{100}$
    \item $(50!)^2$
    \item $100!$
    \item $100^{100}$
\end{enumerate}
To understand this, we note that each of these expressions can be written as the product of 100 terms. The first expression, $2^{100}$, is the smallest because each term is only 2. The second expression, $(50!)^2$, is the product of the numbers 1 through 50 and then 1 through 50 again. Every term is greater than those appearing in the first expression except for the two ones. The third expression, $100!$, is the product of the numbers 1 through 100. The first 50 terms are the same as those in the second expression, but after that the terms are strictly larger. The fourth expression, $100^{100}$, is the product of one hundred hundreds, so every term is greater than all but one of the terms in expression three.



\section{}

\subsection{Problem}
Find the highest power of 5 that divides $1000!$.

\subsection{Solution}
Since $1000!$ is the product of an ascending sequence of numbers and every $5^\text{th}$ term is a multiple of 5, every $5^\text{th}$ term adds a power to our divisor. Similarly, every $25^\text{th}$ term is a multiple of 5 twice, so these terms add 2 to the power of our divisor. The same applies for multiples of $125$ adding 3 to the power and multiples of $625$ adding 4 to the power. Since we have $\frac{1000}{5}$ multiples of 5, $\frac{1000}{25}$ multiples of 25, $\frac{1000}{125}$ multiples of 125, and $\floor{\frac{1000}{625}}$ multiples of 625, we find that our power should be $\frac{1000}{5} + \frac{1000}{25} + \frac{1000}{125} + \floor{\frac{1000}{625}} = 200 + 40 + 8 + 1 = 249$ and the highest power of 5 that divides $1000!$ is $\boxed{5^{249}}$.



\section{}

\subsection{Problem}
Find all $x, y, z \in \Z$ such that $x! + y! = z!$.

\subsection{Solution}




\end{document}
